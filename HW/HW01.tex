\documentclass{article}
\usepackage[utf8]{inputenc}
\usepackage{amsmath,amssymb,amscd,amsfonts,amsthm}

\title{Polymer Physics HW1}
\author{Robert Chang}
\date{20160321}
\begin{document}

%\maketitle



\section*{6.} 
\begin{enumerate}
    \item[(a)] The persistence length ($l_p$) means the flexibility of a polymer chain. For a polymer chain with large persistence length means the chain is high stiffness.
    
    \item[(b)] We know the persistence length is ca. $l_p=6\,nm$, so the Kuhn length of PF8 is ca. $b=2l_p=12\,nm$
    
    \item[(c)] For a PF8 polymer, number of monomer is $N=200$, and length of each monomer is ca. $l=0.7\,nm$. So $r_{max} = Nl = 140\,nm$, and end-to-end distance 
    \[
    <r^2> = 2 l_p r_{max} - 2 l_p^2 \left[ 1- \exp \left(-\frac{r_{max}}{l_p} \right)\right] 
    = 1608\, nm^2
    \]
    \item[(d)] We know that (from lecture) the gyration radius of a Gaussian-like chain, 
    \[
    <R_g^2> = \frac{1}{6} N' b^2,
    \]
    and for rod-like chain,  
    \[
    <R_g^2> = \frac{1}{12} L^2.
    \]
    \begin{itemize}
        \item Case $r_{max} >> l_p$: \\
            $l_p / r_{max} << 1$ and $(l_p / r_{max} )^{n+1}<< (l_p / r_{max} )^{n}$ for  $n >0$.
            So             \[
                \begin{aligned}
                    \frac{ <R_g^2 >}{ r_{max}^2 } 
                    &= \frac{1}{3} \left( \frac{l_p}{r_{max}} \right)
                   -            \left( \frac{l_p}{r_{max}} \right)^2
                   +2           \left( \frac{l_p}{r_{max}} \right)^3
                   -2           \left( \frac{l_p}{r_{max}} \right)^4 \left[ 1-\exp\left( - \frac{r_{max}}{l_p} \right)  \right]\\
               &= \frac{1}{3} \left( \frac{l_p}{r_{max}} \right) 
                \end{aligned}
            \]
            Therefore the gyration radius of the Gaussian-like chain
            \[
                <R_g^2> = \frac{1}{3} ( l_p r_{max} )= \frac{1}{3} N'b \left( \frac{b}{2} \right) = \frac{1}{6} N' b^2
            \]

        \item Case $l_p >> r_{max}$: \\
            Set $x=r_{max} / l_p<< 1$, and we know $\exp (-x) \approx 1 - x + x^2/2 - x^3/6 + x^4 / 24 + O(x^5)$ for $x<<1$.
            So, 
            \[
                \begin{aligned}
                    \frac{ <R_g^2 >}{ r_{max}^2 } 
                    &= \frac{1}{3} \left( \frac{l_p}{r_{max}} \right)
                   -            \left( \frac{l_p}{r_{max}} \right)^2
                   +2           \left( \frac{l_p}{r_{max}} \right)^3
                   -2           \left( \frac{l_p}{r_{max}} \right)^4 \left[ 1-\exp\left( - \frac{r_{max}}{l_p} \right)  \right] \\
                   &= \frac{1}{3x} -  \frac{1}{x^2} + \frac{2}{x^3} - \frac{2}{x^4} \left( x - x^2/2 + x^3/6 - x^4 / 24 + O(x^5)\right) \\
                   &=\frac{1}{12} \qquad \text{(if $x << 1$)}
                \end{aligned}
            \]
            Therefore the gyration radius of a rod-like chain,
            \[
                <R_g^2> =  \frac{1}{12} r_{max}^2 = \frac{1}{12} (N'b)^2 = \frac{L^2}{12}
            \]

    \end{itemize}
 
    
\end{enumerate}



\end{document}

