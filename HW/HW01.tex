\documentclass[a4paper]{article}
\usepackage{amsmath,amssymb,amscd,amsfonts,amsthm}
\usepackage[margin=1in]{geometry}
\usepackage{ccfonts,eulervm} 
\usepackage[T1]{fontenc}

%
%----head and foot setting----
%
\usepackage{fancyhdr}
\pagestyle{fancy}
\renewcommand{\headrulewidth}{0.4pt}
\renewcommand{\footrulewidth}{0.4pt}
\lhead{\today}
\chead{Polymer Physics -- \jobname} %HW01
\rhead{Robert Chang}
%
%
%

\begin{document}

\subsection*{1.}

\subsection*{2.}
We know that for a poly-cis-isoprenece chain, length of monomer unit is $l=0.46\,nm$, 
and end-to-end distance $<r^2> = 16.2\,N$, where $N$ is number of monomer unit. \\
For a ideal chain, $<r^2> = (1+\cos\theta )/ (1-\cos \theta) \, N l^2 =  C_\infty N l^2 = N' b^2$, 
so $C_\infty = 16.2/0.46^2 = 76.56$, and $\cos \theta = 0.974, \theta \approx 13.1^o$.\\
Kuhn length $b = <r^2> / r_{\max} = (16.2N) / (N l \cos (\theta/2)) = 35.47\,nm$.  \\
Kuhn monomer $N' = <r^2> / b^2 = 16.2\,N / 35.47^2 = 1.29 \times 10^{-2}\,N$

\subsection*{3.}
We know the molecular weight of polyethylene is $10^7\, g/mol$, length of monomer $l =0.154\,nm$, and assume $c_\infty = 7.4$.
So the monomer number $N = 10^7/28 = 3.6 \times 10^5$, and end-to-end distance $<r^2> = C_\infty N l^2 = 6.32 \times 10^4\,nm^2$.  \\
From $C_\infty = 7.4 = (1+\cos \theta) / (1-\cos\theta)$, $\cos(\theta) = 0.762, \cos(\theta/2) =0.939$.  \\
The contour length $r_{\max} = N l \cos(\theta /2) = 5.2 \times 10^4\,nm$.
Comparing $<r^2>$ and $r_{\max}$, $b = <r^2> / r_{\max} = 1.22\,nm$ is the Kuhn length of this polymer.

\subsection*{4.}
we know (from lecture) that the end-to-end distance of a freely-rotating ideal chain is 
\[
    <r^2> = N l^2 \left( \frac{1+\cos \theta}{1-\cos \theta} \right) 
\]
So the Kuhn length 
\[
    b = \frac{<r^2>}{r_{\max}} = \frac{N l^2 \cos^2 (\theta /2) / \sin^2 (\theta /2)}{Nl\cos(\theta/2)}
      = l\, \frac{\cos (\theta /2)}{\sin^2 (\theta /2)}
\]
and 
Kuhn monomer 
\[
    N' = \frac{r_{\max}}{b} = \frac{N l \cos (\theta /2)}{l \cos (\theta /2) / \sin^2(\theta /2)}
       = N\, \sin^2(\theta /2)
\]
%
\subsection*{5.}
For an ideal chain, the end-to-end distance
\[
    <r^2> = N l^2 \left( \frac{1+\cos \theta}{1-\cos \theta} \right) \left(\frac{1+<\cos \phi>}{1-<\cos \phi> }\right)
\]
Take log,
\[
    \log(<r^2>) = \log[\, (N l^2)(1+\cos \theta)/(1-\cos \theta)\,]
        + \log(1+<\cos\phi>) - \log (1-<\cos\phi>),
\]
where $<\cos\phi>$ contains temperature effect.
So the variation in temperature $T$ becomes
\[
\frac{ \log(<r^2>)}{dT} = \frac{1}{1+<\cos \phi>} \frac{d <\cos \phi>}{d T} + \frac{1}{1-<\cos \phi>}\frac{d <\cos \phi>}{d T}
= \frac{2}{1-<\cos\phi>^2} \frac{d<\cos \phi>}{dT}
\]
%By symmetry, $\phi \in (0,\,\pi)$. Since $\exp(-U(\phi)/k_BT)$ in trans ($\phi \approx 0$) is much larger than that of in cis ($\theta \approx \pi$). So we have $<\cos \phi> \: > \: <\cos \phi>|_{\text{near trans, $\phi$ from $0$ to $\pi/3$} } > 0$, and $<\cos\phi>^2 \in (0,\,1)$.
%\[
%    \frac{d<\cos \phi>}{dT} = d/dT \,\int_0^{\pi} \cos\phi \exp(-U(\phi)/k_BT) d\phi = \int_0^{\pi} \cos\phi \exp(-U(\phi)/k_BT) U(\phi) /(k_BT^2) d\phi > 0 (<0?)
%\]
%without normalization. 
First we know $<\cos\phi>^2 \in (0,\,1)$, and as $T$ increases, more and more polymer can in the gauge state, implying $\phi$ increases.
Thus the temperature coefficient $d\log(r^2)/dT < 0$.



\subsection*{6.}
    
\begin{enumerate}
    \item[(a)] The persistence length ($l_p$) means the flexibility of a polymer chain. For a polymer chain with large persistence length means the chain is high stiffness.
    
    \item[(b)] We know the persistence length is ca. $l_p=6\,nm$. So under the worm-like chain assumption, the Kuhn length of PF8  is ca. $b=2l_p=12\,nm$
    
    \item[(c)] For a PF8 polymer, number of monomer is $N=200$, and length of each monomer is ca. $l=0.7\,nm$. So under the worm-like chain assumption, $\theta$ is very small, $r_{max} \approx  Nl = 140\,nm$, and end-to-end distance 
    \[
    <r^2> = 2 l_p r_{max} - 2 l_p^2 \left[ 1- \exp \left(-\frac{r_{max}}{l_p} \right)\right] 
    = 1608\, nm^2
    \]
    \item[(d)] We know that (from lecture) the gyration radius of a Gaussian-like chain, 
    \[
    <R_g^2> = \frac{1}{6} N' b^2,
    \]
    and for a rod-like chain,  
    \[
    <R_g^2> = \frac{1}{12} L^2.
    \]
    \begin{itemize}
        \item Case $r_{max} \gg  l_p$: \\
            $l_p / r_{max} \ll  1$ and $(l_p / r_{max} )^{n+1}\ll  (l_p / r_{max} )^{n}$ for  $n >0$.
            So             \[
                \begin{aligned}
                    \frac{ <R_g^2 >}{ r_{max}^2 } 
                    &= \frac{1}{3} \left( \frac{l_p}{r_{max}} \right)
                   -            \left( \frac{l_p}{r_{max}} \right)^2
                   +2           \left( \frac{l_p}{r_{max}} \right)^3
                   -2           \left( \frac{l_p}{r_{max}} \right)^4 \left[ 1-\exp\left( - \frac{r_{max}}{l_p} \right)  \right]\\
               &= \frac{1}{3} \left( \frac{l_p}{r_{max}} \right) 
                \end{aligned}
            \]
            Therefore the gyration radius of the Gaussian-like (linear ideal chain) chain
            \[
                <R_g^2> = \frac{1}{3} ( l_p r_{max} )= \frac{1}{3} N'b \left( \frac{b}{2} \right) = \frac{1}{6} N' b^2
            \]

        \item Case $l_p \gg  r_{max}$: \\
            Set $x=r_{max} / l_p\ll  1$, and we know $\exp (-x) \approx 1 - x + x^2/2 - x^3/6 + x^4 / 24 + O(x^5)$ for $|x|\ll 1$.
            So, 
            \[
                \begin{aligned}
                    \frac{ <R_g^2 >}{ r_{max}^2 } 
                    &= \frac{1}{3} \left( \frac{l_p}{r_{max}} \right)
                   -            \left( \frac{l_p}{r_{max}} \right)^2
                   +2           \left( \frac{l_p}{r_{max}} \right)^3
                   -2           \left( \frac{l_p}{r_{max}} \right)^4 \left[ 1-\exp\left( - \frac{r_{max}}{l_p} \right)  \right] \\
                   &= \frac{1}{3x} -  \frac{1}{x^2} + \frac{2}{x^3} - \frac{2}{x^4} \left( x - x^2/2 + x^3/6 - x^4 / 24 + O(x^5)\right) \\
                   &=\frac{1}{12} \qquad \text{(if $|x| \ll  1$)}
                \end{aligned}
            \]
            Therefore the gyration radius of a rod-like chain,
            \[
                <R_g^2> =  \frac{1}{12} r_{max}^2 = \frac{1}{12} (N'b)^2 = \frac{L^2}{12}
            \]

    \end{itemize}
 
    
\end{enumerate}


\end{document}

