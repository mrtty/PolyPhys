\documentclass{article}
\usepackage[utf8]{inputenc}

\title{PolyPhys HW1}
\author{Robert Chang}
\date{March 2016}
\usepackage{amsmath}

\begin{document}

\maketitle

\section*{6.} 
\begin{enumerate}
    \item[(a)] The persistence length ($l_p$) means the flexibility of a polymer chain. For a polymer chain with large persistence length means the chain is high stiffness.
    
    \item[(b)] We know the persistence length is ca. $l_p=6\,nm$, so the Kuhn length of PF8 is ca. $b=2l_p=12\,nm$
    
    \item[(c)] For a PF8 polymer, number of monomer is $N=200$, and length of each monomer is ca. $l=0.7\,nm$. So $r_{max} = Nl = 140\,nm$, and end-to-end distance 
    \[
    <r^2> = 2 l_p r_{max} - 2 l_p^2 \left[ 1- \exp \left(-\frac{r_{max}}{l_p} \right)\right] 
    = 1608\, nm^2
    \]
    \item[(d)] We know that (from lecture) the gyration radius Gaussian-like chain, 
    \[
    <R_g^2> = \frac{1}{6} N' b^2,
    \]
    and for rod-like chain,  
    \[
    <R_g^2> = \frac{1}{12} L^2.
    \]
    \begin{itemize}
        \item Case $r_{max} >> l_p$: \\
        $l_p / r_{max} << 1$ and $(l_p / r_{max} )^{n+1}<< (l_p / r_{max} )^{n}$ for  $n >0$.
        So
    \[
    \begin{align*}
         \\
    \end{align*}
    \]
        
        
        \item Case $l_p >> r_{max}$:
        
        
        
    \end{itemize}
 
    
\end{enumerate}



\end{document}

